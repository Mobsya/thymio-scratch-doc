\chap{Programming with Events}\label{ch.events}

\sect{Events}

\scrprj{tap-on-off}

In VPL an instruction consists of an event block followed by one or
more action blocks:

\gr{first}{.3}

\emph{Each time} the event occurs, the actions associated with the event
are performed. Here, when the front button of the robot is touched, the
top light is turned on with color red, and when the back button is
touched, the light becomes blue.

In Scratch, the construction equivalent to an event-actions pair is the
\emph{script}:

\gr{tap-on-off}{.3}

The image shows three scripts, each of which changes the costume of the
sprite:

\begin{itemize}

\item Scratch programs are started by clicking on the \emph{green flag}
above the stage. When the green flag is clicked, the costume of the
sprite is set to the \p{blank} costume, which displays the Thymio with
the top lights off.

\item The second script responds to the event of pressing the \bu{r}
key. The costume is changed to the \p{red} costume which displays the
top lights in red.

\item Similarly, the third script turns the lights blue when the \bu{b}
key is pressed.

\end{itemize}

The \bu{Events} palette contains the event blocks which are colored
brown. Most of these blocks have a ``hat'' shape to indicate that they
can only be the first block in a script. After an initial event block,
other blocks can be added to the script.


\sect{Costumes}

A sprite can have one or more \emph{costumes} which specify how the
sprite is displayed on the stage. The sprite's costumes are displayed
when you click the \bu{Costumes} tab. You can create costumes by
importing images, or by drawing or editing them using a paint program
that is included in Scratch.

\informationbox{The costumes for this project}{The project
\bu{thymio-costumes} contains all the costumes used in these projects.
Open this project and copy (drag and drop) the costumes you need to your
\bu{Backpack}. Now open a new project and copy the costumes to the
sprites.}

\informationbox{Setting the size of the costumes}{You may have to adjust
the size of the image of the sprites in order for a project to work. You
can do this by shrinking a costume in the paint program: click on the
icon \scrblk{shrink} and then on the image. A better solution is to use
the block \scrblk[-10]{set-size} from the \bu{Looks} palette in the initialization
of a program; experiment with different values until you have the size the need.}

\sect{Sending and receiving messages}

\scrprj{colors}

Let us look now at the program \bu{colors} which changes the color of
the top lights according to which button is touched
(Figure~\ref{fig.colors}). In Scratch, touching a button is simulated by
clicking on the image of a button. The click is interpreted by a second
sprite called \p{Pointer}, which sends a message to the \p{Thymio} sprite,
depending on which image is clicked: \p{center}, \p{front}, \p{back},
\p{left}, \p{right}.

\begin{figure}
\gr{colors}{.25}
\caption{Changing colors using messages}\label{fig.colors}
\end{figure}

Click on the \p{Pointer} sprite in the sprite area in the lower left of
the Scratch window. Without going into detail, just notice that the
second script uses the block \scrblk[-8]{broadcast-and-wait}, which causes
a message (here, \p{front}) to be sent. Now, click back on the \p{Thymio}
sprite and you will see the scripts in Figure~\ref{fig.colors}. Scripts
starting with the event block \scrblk{when-i-receive} are performed when
the corresponding message is received. We see that different messages
cause the costume of the \p{Thymio} to be changed to ones displaying
different colors.

\sect{Sounds}

\scrprj{bells}

The VPL project \bu{bells} caused notes to be played when the buttons
are touched. In Scratch, the \bu{Sound} palette contains a rich set of
blocks for playing notes. Experiment with these blocks and then replace
the blocks that change costumes in the above project with blocks that
play different sounds when different buttons are clicked. You can also
add sound blocks to the script in Figure~\ref{fig.colors}, so that the
sprite both changes colors and plays sounds.
