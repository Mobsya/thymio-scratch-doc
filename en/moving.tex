\chap{Moving Sprites}\label{ch.moving}

\scrprj{moving}

In the project \bu{moving}, the robot moves forwards and backwards when
the front and back buttons are touched, and its stops when the center
button is touched. In Scratch, a sprite will move on the screen in
response to running blocks from the \bu{Motion} palette.

The following script initializes the sprite to start at the left
side of the stage, pointing right ($90^{\circ}$):

\gr{moving1}{.3}

The size of the stage is from $-240$ to $240$ pixels (picture elements)
in the $x$ (horizontal) direction and $-180$ to $180$ pixels in the $y$
(vertical) direction. You can explore these values by moving your mouse
around the stage; its position is displayed below the lower right corner
of the stage. The initial position of the \p{Thymio} sprite is set to
$(-100,0)$, so that it is at the left edge of the stage and centered
vertically.

The sprite runs the following script when it receives the message
\p{front} that is broadcast when the front button is clicked:

\gr{moving2}{.3}

The block \scrblk{move} is contained within a
\scrblk[-10]{forever} block that causes the instruction to be run
repeatedly. This will cause the sprite to move steadily in the direction
it is pointing.

The \scrblk{wait03} block causes the program to wait 0.3 seconds before
continuing with the next block. This slows down the movement and makes
it easier to click on a button.

The script for responding to the \p{back} message is the same as that
for the front message, except that the sprite is made to point left
($-90^{\circ}$):

\gr{moving3}{.3}

When the \p{center} message is received, all the scripts in the
program are stopped:

\gr{moving4}{.3}

\warningbox{Turning the sprite to move in the direction $-90^{\circ}$
will also cause the \emph{image} of the sprite to turn $180^{\circ}$. To
prevent this, select the \p{Thymio} sprite icon in the area below the
stage and click on the small \bu{i} in the icon. Then click on the dot
which is one of the options in \bu{rotation style}.}

\sect{Directions}

\scrprj{obeys}

The \p{Thymio} can behave like a pet, following you around. In project
\bu{obeys}, click the mouse closer and closer to the center sensor on
the image of the \p{Thymio} sprite. When it is close enough, the red light
on the sensor turns on and the robot moves towards the pointer, stopping
when it is close.

Initialize the project as in \bu{moving} adding the block
\scrblk{switch-costume-to-blank}. The main script for this project is
shown in \cref{fig.move-to}. The script starts running when it receives
the message \p{Come}, which is sent by the \p{Pointer} sprite when the mouse
is clicked.

\begin{figure}
\gr{obeys}{.75}
\caption{Move towards an object near the center sensor}\label{fig.move-to}
\end{figure}

\subsection*{Distance to a sprite}

The robot moves only when it detects that the pointer is close. The
sprite \p{Pointer} tracks the mouse pointer. The block
\scrblk[-4]{distance-to-pointer} returns the distance in pixels from the
sprite that runs it---here the \p{Thymio} sprite---to the \p{Pointer} sprite.
The block can be found in the \bu{Sensing} palette.

Next, we compare this distance with the distance $250$ that we decide is
``close'' enough to be detected. In the palette \bu{Operators} you can
find mathematical operators for performing comparisons:
\scrblk{distance-less-than}.

The block with comparison operator is a \emph{condition block}. The
\p{if} block \scrblk[-15]{if} has an angular area for a condition block
and a ``mouth'' in which you can place other blocks. The meaning of the
\p{if}-block is:

\begin{footnotesize}
\begin{verbatim}
if the condition is true then
    the blocks contained in the "mouth" are run.
\end{verbatim}
\end{footnotesize}

It follows that the rest of the blocks in the script in
\cref{fig.move-to} will be run only when the \p{Pointer} sprite is close to
the \p{Thymio} sprite.

\subsection*{Direction to a sprite}

The specification requires that the Thymio respond only if the pointer
is detected in front of the center sensor. This is implemented by
calling the block \scrblk[-6]{get-pointer-direction-block} which returns in
\scrblk[-4]{direction-to-pointer} the direction of the \p{Pointer} from the
front of the \p{Thymio} sprite. Its value is in degrees in the range 0 to
360, clockwise; that is, directions just to the right of the \p{Thymio} will
have low values, while directions just to the left of the \p{Thymio} will
have values close to $360$. We decide that the \p{Pointer} is detected by
the center sensor if the direction is greater than $325$ or less than $25$.

\informationbox{User-defined
blocks}{\scrblk[-6]{get-pointer-direction-block} and
\scrblk[-4]{direction-to-pointer} are not predefined in Scratch. The
implementation of these blocks is described in \cref{ch.implementation},
but you can use the blocks without understanding the implementation.}

A second \p{if}-block is used to run blocks only when the direction to
the \p{Pointer} is greater than $325$ or less than $25$. The \bu{Operators}
palette contains the operator \scrblk{or} which enables us to combine
the two conditions into one. The result is a \emph{compound condition}
that is true if either of its parts is true.

If the compound condition is true, we turn the \p{Thymio} to face the
\p{Pointer} sprite using the block \scrblk[-6]{point-towards-pointer} from the
\bu{Motions} palette, and we change the costume so that the red light
next to the center sensor is turned on.


\subsection*{Approaching a sprite}

If the \p{Thymio} sprite is close to the \p{Pointer} sprite and is pointed in
the right direction, the robot must approach the sprite. We use
\scrblk[-15]{repeat-until} which is similar to the \p{forever} block in
that it causes the blocks contained in its ``mouth'' to be run again and
again, but it is different in that it has a condition like an
\p{if}-block. The movement of the \p{Thymio} sprite must stop if the
distance to the \p{Pointer} sprite is too small and this is checked by
the condition block \scrblk[-8]{distance-to-pointer-60}.
