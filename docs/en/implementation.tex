\chap{Implementation}\label{ch.implementation}

This Appendix explains advanced programming techniques in Scratch that
were used in the simulations of the Thymio robot. The details are
encapsulated in sprites and new blocks so that the student need not be
concerned with them, but the Appendix can serve are a tutorial to these
programming techniques.

\sect{The {\ttfamily Pointer} sprite}

The \emph{touching} blocks from the \bu{Sensing} palette are used to
detect events. They are condition blocks that return true if the
\textbf{\emph{sprite running the script where the block appears}}
touches something else.

\begin{itemize}
\item \scrblk{touching} returns true if the sprite is touching another
sprite or the mouse pointer.

\item \scrblk{touching-color} returns true if the sprite is touching an area
with this color.

\item \scrblk{color-touching-color} returns true if an area
\textbf{\emph{on this sprite}} with the first color is touching an area
with the second color.

\end{itemize}

\trickbox{To set the color, click the little square,
move the mouse pointer and click an area on a sprite or the
backdrop that has the color you want.}

Scratch does not support checking if the mouse pointer is touching
a color; therefore, we define a sprite called \p{Pointer} that tracks
the location of the mouse pointer (\cref{fig.track}). The costume of
this sprite is a very small gray dot that the user will not see.

\begin{figure}[htb]
\gr{go-to-mouse-pointer}{.25}
\caption{Tracking the mouse pointer}\label{fig.track}
\end{figure}

To simulate the buttons on the Thymio robot, the image of the \p{Thymio}
sprite has different colors for each button. When the \p{Pointer} sprite
is clicked, the script in \cref{fig.buttons} checks if it is touching
one of the colors and broadcasts the appropriate message.

\begin{figure}[htb]
\gr{click-on-buttons}{.3}
\caption{Sensing a click on the buttons}\label{fig.buttons}
\end{figure}


\sect{Computing the direction to the {\ttfamily Pointer} sprite}

Many projects require that the direction from the \p{Thymio} sprite to
the \p{Pointer} sprite be determined. The implementation of the block
\scrblk[-6]{get-pointer-direction-block} is shown in \cref{fig.get-dir}.
The variable \scrblk[-3]{save-dir} is used internally
by this block, and the variable \scrblk[-3]{direction-to-pointer} is
used to return the direction to the calling sprite. Extensive use is
made of the block \scrblk[-3]{direction} which is predefined in the
\bu{Motion} palette and gives the current direction in which the sprite
is pointing.

\begin{figure}[htb]
\gr{get-pointer-direction}{.5}
\caption{Get the direction to the \p{Pointer} sprite}\label{fig.get-dir}
\end{figure}

The \emph{current direction} in which the sprite is pointing is saved in
\scrblk[-3]{save-dir}. Then, the \p{Thymio} sprite is turned so that it
points to the \p{Pointer} sprite. The block \scrblk[-3]{direction} now
contains this new direction. By subtracting the value in
\scrblk[-3]{save-dir} we get the difference between the two directions. The
\p{if}-block ensures that this direction is positive between 0 and 360.
Finally, we turn the \p{Thymio} sprite to point in its original
direction which was stored in \scrblk[-3]{save-dir}.


\sect{Simulating the botton sensors}

To simuate the bottom sensors, create a new costume for the \p{Thymio}
sprite with light-colored areas projecting out in front of the image:

\gr{ground}{.25}

Now declare a variable \scrblk[-4]{is-touching} and define the block
\scrblk[-6]{get-touching-block} (\cref{fig.get-touching}), which returns in
\scrblk[-4]{is-touching} one of \p{both}, \p{left}, \p{right}, \p{neither}
depending on which projecting area is touching the black area on the
stage.

\begin{figure}
\gr{get-touching}{.6}
\caption{Simulation of the ground sensors}\label{fig.get-touching}
\end{figure}