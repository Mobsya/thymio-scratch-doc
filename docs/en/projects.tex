\chap{Guided Projects}\label{ch.projects}

These projects introduce no new concepts so we present them as a guided
projects.

\sect{Line following}

\scrprj{follow-line}

Construct a backdrop with a wide line that represesnts a strip of black
tape on the floor:

\gr{line}{.2}

Next, import the costume called \p{ground}:

\gr{ground}{.2}

This costume has two small areas projecting from the front of the robot
and is used to implement the block \scrblk[-6]{get-touching-block} as
described in \cref{ch.implementation}. The block returns in the variable
\scrblk{is-touching} one of \p{both}, \p{left}, \p{right}, \p{neither}
depending on which projecting area is touching the black area on the
stage. The \p{Thymio} sprite needs one script whose algorithm is shown
in \cref{fig.line-following}.

\begin{figure}
\begin{footnotesize}
\begin{verbatim}
initialize the position of the Thymio
forever
    call get-touching-block
    if is-touching = both
        move forward
    else
        if is-touching = right
            turn right
        else
            if is-touching = left
                turn left
            else
                stop the script
\end{verbatim}
\end{footnotesize}
\caption{Algorithm for line following}\label{fig.line-following}
\end{figure}

\subsection*{Real robots don't move straight}

From your experience with the Thymio robot, you know that it will not
move straight even if both motors are set to the same power. The wheels
and motors are not identical, the friction may vary, and so on. This
noisy behavior is called \emph{jitter}. Place the following block before
first \p{move} block to simulate jitter:

\gr{jitter}{.4}

The block causes the robot to turn by an angle that is randomly chosen
in the range $-20$ to $20$. You will see that the robot moves very
erratically, but the algorithm succeeds in keeping the robot on the line.


\sect{Sweeping the floor}

\scrprj{sweep}

We want the \p{Thymio} sprite to traverse the stage so as to cover the
whole surface:

\gr{sweep-stage}{.3}

In order the evaluate the quality of the sweep, the robot draws a thin
red line as it moves. \scrblk{pen-down} causes an imaginary pen in the
sprite to be lowered ``down'' so that it is in contact with the stage.
The pen color is set to red.

The simulation could be programmed simply by commanding the
sprite to move the correct distances---$280$ steps horizontally and
$60$ steps vertically---turning right and left by $90^{\circ}$ as
required. However, there is no way to command the Thymio \emph{robot} to
travel a certain distance. All we can do is to set the motors to a
constant power for a fixed time, but the speed will not be precisely
constant for the reasons mentioned above, so the distance will not be
predictable.

\begin{figure}
\gr{sweep1}{.35}
\caption{Sweeping the stage}\label{fig.sweep}
\end{figure}

The script shown in \cref{fig.sweep} causes the sprite to move 4 steps
at a time. A second script (not shown) consists of a
\scrblk[-15]{repeat-until} block that contains blocks for left and right
turns of $90^{\circ}$ with \scrblk{wait03} blocks between them. Adjust
the durations in the \p{wait} blocks to obtain a good sweep of the
stage.

\sect{Finite automata}

\scrprj{finite-automaton}

The finite automata project in the VPL tutorial was inspired by the
light-painting projects described on the Thymio website. The idea is to
have the robot sweep over an area and respond to codes placed on the
floor. The code consists of sync (synchronization) marks and marks of symbols
on the stage. \cref{fig.fa} shows the stage with 4 rows of 4 sync marks
(the blue circles) and 9 vertical black lines representing the symbol 1.

\begin{figure}[htb]
\gr{fa-stage}{.3}
\caption{Synchronization and symbol marks}\label{fig.fa}
\end{figure}

\newpage

The \p{Thymio} sprite is required to sweep the stage and record the number
of 1's. (For a finite automata we need to specify a finite range, such
whether the number of 1 symbols is even or odd. This can be done using
the \p{mod} operator.)

The sprite can be in three states: (1) it is touching a sync mark alone,
(2) it is touching both a sync mark and the black line encoding 1, and
(3) it is touching neither. Use \scrblk{touching-color-blue} to detect
the blue ball representing the sync marks and
\scrblk{touching-color-black} to detect black line representing 1.

To check the behavior of the sprite,
display different colored top lights in each state.

\warningbox{The sync marks should be sufficiently far apart so that
the \p{Thymio} sprite doesn't touch two marks at the same time.}

\subsection*{Synchronizing the sprites}

There will be two sprites: the \p{Thymio} sprite that sweeps the floor
and decodes the marks, and a \p{ball} sprite that stamps marks on the
stage before the \p{Thymio} sprite starts to move. Use messages to
synchronize the sprites:

\begin{itemize}
\item The \p{Thymio} sprite performs initialization and then
broadcasts the message \p{draw}.
\item When the \p{ball} sprite receives the \p{draw} message, it stamps the
marks and then broadcasts the message \p{go}.
\item When the other scripts in the \p{Thymio} sprite receive the
\p{do} message, they begin the tasks of sweeping the stage and decoding
the marks.
\end{itemize}

\sect{Timed behavior}

\scrprj{shy}

Specification: If the \p{Pointer} is clicked opposite the right sensor,
the \p{Thymio} sprite turns until the \p{Pointer} is opposite the center
sensor and then 4 seconds later turns back. Similarly, if the
\p{Pointer} is clicked opposite the left sensor. Be sure that the small
red lights on the left, right and center sensors are turned on and off as
appropriate.

Guidance: The behavior for detection by the left and right sensors is
the same except for the costumes. Create a new block
\p{turn-and-turn-back} with one parameter---the direction---that
includes all the blocks needed to implement the behavior of the
\p{Thymio} sprite once the \p{Pointer} sprite is detected.

\newpage

\sect{Catch the mouse}

\scrprj{mouse}

Specification:

\begin{itemize}
\item The \p{Thymio} sprite detects a click of the \p{Pointer} sprite
opposite its left sensor.
\item It turns left (counterclockwise) until the right sensor points
towards the \p{Pointer}.
\item It turns right (clockwise) until the center sensor points towards
the \p{Pointer}.
\item It moves towards the point of the click and then stops.
\end{itemize}

Guidance:

\begin{itemize}

\item Declare a variable that takes four values:
\p{search-left}, \p{search-right}, \p{found}, \p{stop}. This variable
will take on successive values as it accomplishes each part of the
mission.

\item The behavior of the \p{Thymio} sprite will depend on which state
it is in. It is convenient to define new blocks for each state:
\p{go-left}, \p{go-right}, \p{catch-mouse}.

\end{itemize}
