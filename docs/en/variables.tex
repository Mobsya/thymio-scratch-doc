\chap{Variables}\label{ch.variables}

\sect{Using a variable to store a state}

In advanced mode, VPL projects can use \emph{states} to remember
situations and values. There are four elements (``quarters'') to the
state block, so there are 16 different states. Scratch supports
\emph{variables} which can be used to store values. They offer many more
capabilities than VPL states, because there can be as many variables as
we want and each variable can store a large range of values.

\scrprj{tap-on-off-state}

Let us implement the project \bu{tap-on-off} from the tutorial. Clicking
on the \p{Thymio} sprite will turn the top light on if it is off and turn it
off if it is on. A variable will remember the current state \p{on}
or \p{off}. First, we \emph{declare} the variable. Go to the \bu{Data}
palette and click \scrblk{make-variable}; give the variable a name that
makes clear what its purpose is. An oval block with
the variable's name appears, together with some new blocks:

\gr{declare}{.3}

\informationbox{Displaying the value of the variable}{The check next
to the variable block is used to indicate that you want the variable and
its current value to be displayed on the stage.}

We need to \emph{initialize} the variable in the script that is
run when the green flag is clicked. This gives the variable a
value before it is used for the first time. Here, the block
\scrblk[-8]{set-block} sets the value to \p{off}:

\gr{tap-on-off-state-init}{.35}

The following script starts with the block \scrblk[-8]{when-sprite-clicked},
meaning that it begins to run when the sprite that contains the script is clicked:
 
\gr{tap-on-off-state}{.4}

It checks the value of the variable \p{state} and decides whether to
switch the costume to the \p{red} costume or to the \p{blank} costume. It also
changes the current value of the variable \p{state} to the opposite
value.


\sect{Using a variable to store a number}

\scrprj{count-to-two}

\scrprj{count-to-four-binary}

\scrprj{add}

Variables are commonly used to store numbers. The VPL project
\bu{count-to-two} encoded the numbers 0 and 1 in states, and later
projects counted to 4 and even to 16. We show how to use variables to
count to two in binary, and leave the extension to other numbers as
exercises.

The Thymio robot displays the current state by lighting the
diagonal segments of the circular lights around the buttons. We simulate
this by displaying an orange circle between the buttons on an image:

\gr{buttons-with-light}{.3}

There are two aspects to the simulation in Scratch: (1) initializing,
incrementing and checking the variable, and (2) displaying and erasing
the orange circle.

We first discuss the operations on the variable. The variable is named
\p{count} and is initialized to zero in the first script:

\gr{count-2-init}{.2}

In the second script, the choice whether to display or erase the circle
is done in an \p{if}-statement that checks if the value of
\scrblk{count-block} is 0 or not, using the \scrblk[-20]{if-else} block:

\gr{count-2}{.35}

If the condition is true, the blocks in the first ``mouth'' are run; if
not, the blocks in the second ``mouth'' are run.

The actual display of the circle is done within the two blocks
\scrblk{display-block} and \scrblk{erase-block} whose definition is
given below.

The final block of the script is \scrblk{change} which changes the
(numerical) value in the variable by the amount written in the small
square. Here, \scrblk{mod} adds 1 to the value of \scrblk{count-block}
and then takes the remainder (called \p{mod}) by 2. The result will be
either 0 or 1.

\warningbox{Be sure not to confuse \scrblk{set-one} with
\scrblk{change}. The first block ignores the current value in the
variable and sets the variable to the value written in the small square.
The second block takes the current value of the variable, adds the
value written in the small square and then sets the value of the
variable to the result of the computation. To subtract a value
from the current value, simply change by a negative number.}

\sect{Drawing on the stage}

Scratch supports drawing on the stage using blocks available in the
\bu{Pen} palette. We will not explain all the blocks here, just the ones
used to display and erase the orange circle.

In the initialization script, \scrblk{clear} erases existing marks on
the stage , if any. The blocks \scrblk{display-block} and
\scrblk{erase-block} contain \scrblk{stamp} which prints the image of
the current costume of the sprite on the stage:

\gr{display-erase}{.5}

\warningbox{Be sure not to confuse sprites and stamps.
A sprite is an actor in a Scratch animation; it has scripts and
costumes. When it moves on the stage it does not leave a mark.
The block \scrblk{stamp} makes a mark on the stage; the mark is the
current costume of the sprite that runs the block. The stamp is not an
actor and does not move, and the mark remains on the stage
until removed by running \scrblk{clear}.}

Don't use the \scrblk{clear} when you only need to erase one mark such
as an orange circle. Instead, change the costume of the sprite to one
that is all white and stamp it in exactly the same place as the mark of
the orange circle.

The block \scrblk{hide} is used in the initial script because we don't
want to display the sprite, only the marks that it makes on the stage.

\sect{Why two sprites?}

There are two sprites in this project: a \p{Thymio} sprite (well, only
the buttons), and a \p{circle} sprite. The \p{Thymio}
sprite is the one that is clicked on; it broadcasts a message to the
\p{circle} sprite. We don't want to sense a click on the \p{circle}
sprite, because it may not be visible, and we don't want to stamp the
\p{Thymio} sprite because we want marks of the circle, not of the
buttons.

\sect{Parameters}

We want to use the blocks \scrblk{display-block} and
\scrblk{erase-block} in more than one place with different values for
the positions of the marks. The definitions of the blocks
\scrblk[-15]{display-head} and \scrblk[-20]{erase-head} declare two
\emph{parameters} called \p{x} and \p{y}. When the blocks are used, the
x- and y-values of the position need to be provided in the small
squares, just as is done in predefined blocks like \scrblk{go-to-block}.

Within the definition of a block with parameters, the current values of
the parameters are available as oval blocks that can be used like any
other variable.
